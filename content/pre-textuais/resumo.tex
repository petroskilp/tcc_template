% RESUMO--------------------------------------------------------------------------------

\begin{resumo}[RESUMO]
\begin{SingleSpacing}


Resumo na língua vernácula, obrigatório em monografias, teses e dissertações, constituído de uma sequência de frases concisas e objetivas, fornecendo uma visão rápida e clara do conteúdo do estudo.Devem ser seguidas as recomendações da NBR 6028. O resumo informativo é redigido em parágrafo único, espaço simples. Seguido das palavras-chave. Recomendações quanto ao número de palavras: a) de 150 a 500 palavras os de trabalhos acadêmicos (teses, dissertações e outros) e relatórios técnico-cientifícos; b) de 100 a 250 palavras os de artigos de periódicos; c) de 50 a 100 palavras os destinados a indicações breves. Também, não deve conter citações. Usar o verbo na terceira pessoa do singular, com linguagem impessoal, bem como fazer uso, preferencialmente, da voz ativa. Texto contendo um único parágrafo.\\

\textbf{Palavras-chave}: Palavra. Segunda Palavra. Outra palavra.

\end{SingleSpacing}
\end{resumo}

% OBSERVAÇÕES---------------------------------------------------------------------------
% Altere o texto inserindo o Resumo do seu trabalho.
% Escolha de 3 a 5 palavras ou termos que descrevam bem o seu trabalho 

